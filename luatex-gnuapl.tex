\documentclass[12pt,a4paper]{article}
\oddsidemargin 0pt
\textwidth 160mm

\usepackage{fontspec}
\setmonofont[Path=/home/plain/.fonts/]{apl385.ttf}

\title{Executing APL code from inside \TeX}
\author{Dirk Laurie}
\date{14 March 2015}

\newcommand\showapl[1]{\directlua{tex.print(apl.texeval"#1")}}
\directlua{apl=require"gnuapl"}

\begin{document}
\maketitle

The \verb"gnuapl" Lua package allows you to evaluate APL code 
directly in \TeX\ and display it together with its output, without any 
need for externally generated files, as it would appear in a GNU APL 
interactive session. All you need to do is execute
\begin{verbatim}
   lualatex myfile.tex
\end{verbatim}
for a suitable input file containing commands like 
\verb"\showapl{1⌽3 4⍴⍳12}". That command inserts the following output
into \verb"myfile.pdf".

\showapl{1⌽3 4⍴⍳12}

The macro \verb"\showapl" is defined in the preamble of \verb"myfile.tex" 
(see below), which also contains a command to load the package. In this 
document we describe only the \verb"texeval" command of \verb"gnuapl". 
The rest of the package is aimed more at standalone Lua applications and 
is documented elsewhere. 

\subsection*{\texttt{texeval(str[,option]})}
   Pass the APL code in \texttt{str} to the GNU APL interpreter for 
evaluation and optionally include the code and its result in the \TeX\
document.

\begin{description}
\item{\texttt{str}} A string containing APL code, which will be evaluated
by 
\item{\texttt{option}}
   \begin{description}
      \item{\texttt{option=1}} Show the code, indented as in an APL session.
      \item{\texttt{option=2}} Show the result, not indented.
      \item{\texttt{option=3}} Both of the above.
   \end{description}
\end{description}

\pagebreak

\section*{Preamble of this document}

You can use the following preamble in your own \TeX\ source, after
of course customizing some of the fields.

\begin{verbatim}
\documentclass[12pt,a4paper]{article}
\oddsidemargin 0pt
\textwidth 160mm

\usepackage{fontspec}
\setmonofont[Path=/home/plain/.fonts/]{apl385.ttf}

\title{Executing APL code from inside \TeX}
\author{Dirk Laurie}
\date{14 March 2015}

\newcommand\showapl[1]{\directlua{tex.print(apl.texeval"#1")}}
\directlua{apl=require"gnuapl"}
\end{verbatim}

The font \texttt{apl385.ttf} is freely available in many places in 
the Internet. Any other fixed-space font that offers the APL glyphs
should work as well. In my opinion, APL code does not look good in
a variable-space font.

\end{document}
